\documentclass[12pt, utf8]{article}
\usepackage[utf8]{inputenc}
\renewcommand{\familydefault}{\rmdefault}
\usepackage[T1]{fontenc}
\usepackage{geometry}
\usepackage{amsmath} % Advanced math typesetting
\geometry{verbose,tmargin=1in,bmargin=1in,lmargin=0.75in,rmargin=0.75in}
\setcounter{secnumdepth}{2}
\setcounter{tocdepth}{2}
\setlength{\parindent}{0cm}
\usepackage{setspace}
\usepackage{fancyhdr}
\usepackage{mathtools}

\pagestyle{fancy}
\fancyhead[C]{\rule{.5\textwidth}{4\baselineskip}}% Add something BIG in the header
\setlength{\headheight}{15pt}% ...at least 14.5pt
\fancyhf{}
\chead{6643 HW1}
\rhead{\today}
\lhead{Guannan Shen}
\rfoot{Page \thepage}
 

\usepackage{extramarks}
\usepackage{lastpage}
\usepackage{chngpage}
\usepackage{soul}
\usepackage[usenames,dvipsnames]{color}
\usepackage{graphicx,float,wrapfig}
\usepackage{ifthen}
\usepackage{listings}
\usepackage{courier}
\usepackage{framed}

\begin{document}




\newenvironment{problem}{\begin{framed}\begin{bfseries}}{\end{bfseries}\end{framed}\vspace{11pt}}


\begin{problem}

1. Consider a first-order autoregressive process:
  
a. Determine $E(\varepsilon_t)$

b. Determine $Cov(\varepsilon_t , \varepsilon_{t+h})$

c. Determine $Corr(\varepsilon_t , \varepsilon_{t+h})$

d. Is $\varepsilon_t$ a stationary process?
\end{problem}

a.
Since $Z_t \sim N(0, \sigma^2) $:
\begin{eqnarray}
E(\varepsilon_t) &=& E(\sum_{j = 0}^\infty \phi^j Z_{t-j} ) \\
&=& \sum_{j = 0}^\infty [E(\phi^j Z_{t-j} )] \\
&=& \sum_{j = 0}^\infty [\phi^j \times E(Z_{t-j})] \\
&=& E(Z_{t-j}) \times \sum_{j = 0}^\infty \phi^j  \\
&=& 0 \times \sum_{j = 0}^\infty \phi^j \\
&=& 0
\end{eqnarray}
b. Since $ E(Z_t) = 0, Var(Z_t) = \sigma^2$, \\
we have $ E(Z_t^2) = Var(Z_t)^2 + E(Z_t)^2 = \sigma^2 $, \\
we also know that $ \left| \phi \right| < 1 $, \\
and $Z_t$ are \textit{i.i.d}:
\begin{eqnarray}
Cov(\varepsilon_t , \varepsilon_{t+h}) &=& 	E(\varepsilon_t \varepsilon_{t+h}) - E(\varepsilon_t)E(\varepsilon_{t+h}) \\
&=& E(\varepsilon_t \varepsilon_{t+h}) \\
&=& E[(\sum_{j = 0}^\infty \phi^j Z_{t-j}) (\sum_{i = 0}^\infty \phi^i Z_{t+h-i})] \\ 
&=& E[\overbrace{\sum_{j = 0}^\infty \sum_{i = 0}^\infty (\phi^j Z_{t-j} \phi^i Z_{t+h-i})}^{t-j \neq t+h - i} + \overbrace{\sum_{i = 0}^\infty (\phi^{i - h + i} Z_{t+h-i}^2) }^{t-j = t+h -i} ] \\
&=& \overbrace{ E(\sum_{j = 0}^\infty  \phi^j Z_{t-j})E(\sum_{i = 0}^\infty \phi^i Z_{t+h-i})}^{t-j \neq t+h - i} + \overbrace{E (\sum_{i = 0}^\infty \phi^{i - h + i} Z_{t+h-i}^2)}^{t-j = t+h -i} ] \\
&=&  E(Z_{t-j}) \sum_{j = 0}^\infty \phi^j E(Z_{t+h-i}) \sum_{i = 0}^\infty \phi^i + E(Z_{t+h-i}^2)) \sum_{i = 0}^\infty \phi^{i - h + i} \\
&=& 0 \quad + \quad \sigma^2 (\dfrac{\phi^{-h}}{1 - \phi^2}) \\
&=& \phi^{-h}(1- \phi^2)^{-1}\sigma^2
\end{eqnarray}
c. 
\begin{eqnarray}
Corr(\varepsilon_t , \varepsilon_{t+h}) &=& \dfrac{Cov(\varepsilon_t , \varepsilon_{t+h})}{\sqrt{Var(\varepsilon_t)Var(\varepsilon_{t+h})}} \\ 
Var(\varepsilon_t) &=& Cov(\varepsilon_t , \varepsilon_t) \\
&=& (1- \phi^2)^{-1}\sigma^2 \\ 
Var(\varepsilon_{t+h}) &=& Cov(\varepsilon_{t+h}, \varepsilon_{t+h}) \\ 
&=& (1- \phi^2)^{-1}\sigma^2 \\ 
\text{Thus, } Corr(\varepsilon_t , \varepsilon_{t+h}) &=& \dfrac{\phi^{-h}(1- \phi^2)^{-1}\sigma^2}{(1- \phi^2)^{-1}\sigma^2} \\
&=& \phi^{-h}
\end{eqnarray}
d. AR(1) is a weakly stationary process, since the mean and variance is the same for all t and the covariance between $ \varepsilon_t , \varepsilon_{t+h}$ is the same for all t. 

\end{document}